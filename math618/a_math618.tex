\documentclass[12 pt]{article}

\topmargin -0.25in
\textheight 8.75in

\usepackage{amsmath,amsfonts}
\title{Fourier Analysis on Coset Spaces}
\author{Brian Forrest\\presented by Alan Kydd}
\date{April 2, 2004}

%\pagestyle{plain}

\newcommand\done{\begin{flushright}$\Box$\end{flushright}}

\newtheorem{defn}{Definition}
\newtheorem{theorem}[defn]{Theorem}
\newtheorem{prop}[defn]{Proposition}
\newtheorem{cor}[defn]{Corollary}
\newtheorem{lemma}[defn]{Lemma}

\begin{document}

\maketitle

%\begin{slide}
\section{Introduction}

$G$ a locally compact group, $H$ a closed subgroup of $G$.  We define and study
natural analogs of the Fourier and Fourier-Stieltjes algebras for $G/H$, and show
that when $H$ is compact, $A(G/H)$ can be used to study the nature of $G/H$ in a manner
similar to the group case.

%\textbf{Table of Contents}
%\begin{enumerate}
%\item[1.] Defining $A(G/H)$ and $B(G/H)$
%\item[2.] Structure of $A(G/K)$ for compact subgroup $K$
%\item[3.] Weak amenability of $A(G)$
%\end{enumerate}

\section{Defining $A(G/H)$ and $B(G/H)$}
In this section, definitions for $A(G/H)$ and $B(G/H)$ are given, and it is shown
that these definitions are useful analogs of the Fourier and Fourier-Stieljes algebras
for groups.

\bigskip
\noindent\textbf{Notation}\\
Let $H$ be a closed subgroup of $G$.  $q : G \rightarrow G/H$ denotes the canonical quotient map.
$\tilde{x}$ denotes the left coset $xH$.  We have an isomorphism between
$C(G/H)$ and $C(G:H)= \{f\in C(G):f(xh) = f(x)\;\forall x\in G,\;h\in H\}$
via the map $\tilde{f} \mapsto f$, where $f = \tilde{f}\circ q$.
We denote the equivalence class of all continuous unitary
representations of $G$ by $\Sigma_G$.  For $\pi \in \Sigma_G$, we let $A_\pi$ denote the closed linear
span of the coefficient functions of $\pi$, and we denote the weak-* closure of $A_\pi$ by $B_\pi$.
For $\rho$, the left regular representation of $G$ on $L_2 (G)$, $A_\rho$ is usually denoted $A(G)$.

%\pagebreak
\begin{defn}
$$B(G:H) = \{u\in B(G):u(xh) = u(x)\;\forall\;x\in G,\;h \in H\},$$
$$A(G:H) = \{u\in B(G:H): q(\text{supp}\;u)\;\text{compact in}\;G/H\}^{-\|\cdot\|_{B(G)}}.$$
\end{defn}


\pagebreak
%3.1
\begin{prop}
\begin{list}{}{\itemsep -4pt}
\item[(i)] $B(G:H)$,$A(G:H)$ are closed subalgebras of $B(G)$.  Moreover, $A(G:H)$ is a closed ideal in 
$B(G:H)$.
\item[(ii)] $B(G:H)$ is unital.
\item[(iii)] $A(G:H)\cap A(G) \neq \{0\}$ iff $H$ is compact.
\item[(iv)] $A(G:H) = B(G:H)$ iff $G/H$ is compact.
\end{list}
\end{prop}

It is known that $A(G:H)$ is isometrically isomorphic to $A(G/H)$ when $H$ is compact
and normal.  The compactness of $H$ is not necessary:

%3.2
\begin{prop}
Let $H$ be a closed normal subgroup of $G$.  Then $B(G:H)$ and $A(G:H)$ are isometrically
isomorphic to $B(G/H)$ and $A(G/H)$ respectively.
\end{prop}

It is shown below that for compact $K$, $B(G:K)$ and $A(G:K)$ are complemented subspaces
of $B(G)$, $A(G)$ respectively.

%3.3
\begin{theorem}
Let $H$ be a closed subgroup of $G$.  Then there exists a projection
$P:B(G) \rightarrow B(G:H)$ with $\|P\| \leq 1$.
\end{theorem}

In general, $P$ does not map $A(G)$ onto $A(G:H)$.  However, for compact subgroup $K$,
we have
$$
P(f)(x) = P_K (f)(x) := \int_K f(xk) dk.
$$
Thus we have the following corollary:

%3.4
\begin{cor}\label{3.4}
Let $K$ be a compact subgroup of $G$.  Then $P_K$ is a continuous projection of $B(G)$ onto $B(G:K)$.
The restriction of $P_K$ to $A(G)$ is a projection of $A(G)$ onto $A(G:K)$.
\end{cor}
%\end{slide}


%% remarks, pg 178,179
The analog of the Fourier Algebra for the coset space $G/H$ is usually considered to be
the space $A_{\pi_H}$, where $\pi_H$ is the quasi-regular representation of $G$ determined
by $H$.  This definition has two major problems.

\medskip
\noindent{\bf Problem I.} $A_{\pi_H}$ is in general not an algebra.
For example, when $K$ is a compact subgroup, $A_{\pi_H}$ is an algebra iff $A_{\pi_H} = A_{\pi_{K_1}}$, 
where $K_1 = \cap_{x\in G} xKx^{-1}$.  Since $K_1$ is normal, $A_{\pi_{K_1}} = A(G/K_1)$
(Arsac, 1976).  From these results it follows that for $K$ compact, $A(G:K)\neq A_{\pi_H}$
unless $K$ is normal.

A little more can be said:

%3.7
\begin{prop}
Let $K$ be a compact subgroup of $G$.  Then $A(G:K) = A_\pi$ for some $\pi \in \Sigma_G$ iff $K$
is normal.
\end{prop}

\medskip
\noindent{\bf Problem II.} It is possible to have two distinct closed subgroups $H_1$, $H_2$, and yet
$A_{\pi_{H_1}} = A_{\pi_{H_2}}$, even when these subgroups are compact (Arsac, 1976).  However,
we do not have this problem for the space $A(G:K)$ for a compact subgroup $K$:

%3.5
\begin{prop}\label{3.5}
Let $K_1$, $K_2$ be compact subgroups of $G$.  Then $A(G:K_1) = A(G:K_2)$ iff $K_1 = K_2$.
If $G$ is a [SIN]-group and $H_1$, $H_2$ are closed subgroups of $G$ with $H_1 \neq H_2$, then
$A(G:H_1) \neq A(G:H_2)$.
\end{prop}
{\it Proof:}
We give here the proof for the first case only.  Clearly, $K_1 = K_2$ implies that $A(G:K_1) = A(G:K_2)$.
Assume that $x_0 \in K_1$ and $x_0 \not\in K_2$.  Then there is an open $\tilde{U} \subset G/K_2$
with $\tilde{e}\in\tilde{U}$ and $\tilde{x}_0 \not\in \tilde{U}$.  Let $U = q_{K_2}^{-1}(\tilde{U})$.
$U$ is an open neighbourhood of $K_2$ not containing $x_0$.  We can find $u\in A(G)$ such that $u(x) = 1$
for $x\in K_2$ and $u(x)=0$ if $x\not\in U$.  Let $u_1 = P_{K_2}u$.
$u_{1}(xk_2) = \int_{K_2}u(xk_{2}k)\,dk = \int_{K_2}u(xk)\,dk = u_{1}(x)$ for $x\in G$, $k_2 \in K_2$.  That is,
$u_1 \in A(G:K_2)$.  $u_1(e) = 1$ while $u_1(x_0) = 0$, and so $u_1 \not\in A(G:K_1)$.\done

%3.6
\begin{cor}
Let $K_1$, $K_2$ be compact subgroups of $G$.  Then $B(G:K_1) = B(G:K_2)$ iff $K_1 = K_2$.  If $G$
is a [SIN]-group, then $B(G:H_1) = B(G:H_2)$ for $H_1$, $H_2$ closed subgroups of $G$, iff $H_1 = H_2$.
\end{cor}
{\it Proof:}
in either case, $u_1$ as constructed above is in $B(G:K_2)$\;$[B(G:H_2)]$,
but not in $B(G:K_1)$\;$[B(G:H_1)]$.\done


% remarks, pg 179
In light of the above problems for $A_{\pi_H}$, $A(G:H)$ and $B(G:H)$ are more useful analogs for $G/H$ of the
Fourier and  Fourier-Stieltjes algebras.

\begin{defn}
We define $A(G/H)$, the Fourier algebra of the coset space $G/H$, to be the subalgebra
of $C(G/H)$ identified with $A(G:H)$.
\end{defn}

\begin{defn}
We define $B(G/H)$, the Fourier-Stieltjes algebra of the coset space $G/H$, to be
the subalgebra of $C(G/H)$ identified with $B(G:H)$.
%Moreover, we give $B(G/H)$ the obvious norm.
\end{defn}

When $H$ is a compact subgroup, $A(G/H)$ and $B(G/H)$ have many of the same properties
of $A(G)$, $B(G)$.

We have the definitions for the almost periodic and weakly almost periodic
functions on a coset space (Skantharajah, 1985):

\begin{defn}
$AP(G/H)$ is the set of all $f\in C(G/H)$ such that the set $\{ _x f: x\in G\}$ is
relatively compact in the norm topology of $C(G/H)$.
\end{defn}

\begin{defn}
$WAP(G/H)$ is the set of all $f\in C(G/H)$ such that the set $\{ _x f: x\in G\}$ is
relatively compact in the weak topology of $C(G/H)$.
\end{defn}
\pagebreak
%3.8
\begin{prop}
Let $H$ be a closed subgroup of $G$.  Then,
\begin{list}{}{\topsep -4pt \itemsep -4pt}
\item[(i)] $B(G/H)\subseteq WAP(G/H)$, and $B(G/H)\cap AP(G/H)$ is the space identified with
$B(G:H)\cap AP(G)$.
\item[(ii)] $B(G/H)\cap AP(G/H)$ is a complemented subalgebra of $B(G/H)$ with the Radon-Nikodym
property.
\end{list}
\end{prop}
{\it Proof:}
(i) follows from Skantharajah (1985).
(ii): $B(G/H)\cap AP(G/H)$ is an algebra.  $B(G)\cap AP(G)$ has the RNP
(Lahoue, 1973) and is complemented in $B(G)$.  $B(G)\cap AP(G)$ is of the form
$A_\pi$ where $\pi$ is the left regular representation of the almost periodic
compactification of $G$.  For the projection, take $P=P_\pi \circ P_H$, where
$P_\pi$ is the projection determined by $\pi$.\done


%remarks, pg 180
Recall that $A(G)$ is sup-norm dense in $C_0 (G)$.  For a compact subgroup $K$,
$A(G/K)$ is also sup-norm dense in $C_0 (G/K)$.  However, when $H$ is not compact,
there may exist $f\in C_0 (G/H)$ such that $f\not\in WAP(G/H)$ (Chou).  Thus $A(G/H)$ may
not separate the points of $G/H$.  The proof of Proposition~\ref{3.5} gives the following
result.
%3.9
\begin{theorem}
Let $G$ be a [SIN]-group with a closed subgroup $H$.  Then $A(G/H)$ separates points in $G/H$.
\end{theorem}


This next proposition extends a result of Herz (1973).

%3.10
\begin{prop}
Let $K$ be a compact subgroup of $G$.  Let $H$ be a closed subgroup of $G$ such that $K\subseteq H$.
Then every $\tilde{u} \in A(H/K)$ extends to a function $\tilde{u}_1 \in A(G/K)$ with
$\|\tilde{u}\|_{A(H/K)} = \|\tilde{u}_1\|_{A(G/K)}$.
\end{prop}
{\it Proof:}
Let $u\in A(H:K)$ be the function identified with $\tilde{u}$.  By Herz's result,
$u$ extends to some $v\in A(G)$ of equal norm.  Let $u_1 = P_K (v)$.
Since $\|P_K \| \leq 1$, $\tilde{u}_1$ is the desired extension.\done

We can extend a $u \in B(H)$ to some $v\in B(G)$ when $G$ is a [SIN]-group
or if $H$ is normal (Cowling, Rodway, 1979).  Modifying the argument
above produces the following proposition:

%3.11
\begin{prop}
Let $H$ be a closed subgroup of $G$.  Assume that either $G$ is a [SIN]-group or that $H$ is normal.
Let $H_1$ be another closed subgroup containing $H$.  Then every $\tilde{u}\in B(H_1/H)$ extends to a 
$\tilde{u}_1 \in B(G/H)$ with the same norm.
\end{prop}
%{\it Proof:}
%Let $u$ be the function is $B(H_1:H)$ associated with $\tilde{u}$.  If $G$ is a 
%[SIN]-group, then ???\done.


% Section 4
\section{Structure of $A(G/K)$ for compact subgroup $K$}
\textbf{Notation and Definitions}\\
%We assume we have a measure $\mu$ on $G/H$ such that for every $f\in L_1 (G)$,
%$$
%f(x) = \int_{G/H} \int_K f(xk) dk\;d\mu (\tilde{x}).
%$$
Let $\mathcal{A}$ be a semisimple commutative Banach algebra.  $\Delta (\mathcal{A})$ denotes
the maximal ideal space of $\mathcal{A}$.  Given any closed set $A$ of $\Delta (\mathcal{A})$,
we define the following ideals:\\
$I(A) = \{u\in \mathcal{A}: u(x) = 0\;\forall\;x\in A\}$\\
$j(A) = \{u\in I(A): \text{supp}\;u\;\text{is compact} \}$\\
$J(A)=$ the norm closure of $j(A)$ in $I(A)$.\\ 
$A$ is said to be of spectral synthesis if $I(A) = J(A)$.
$A$ is said to be of weak spectral synthesis if for each $u\in I(A)$, there exists a positive integer $n$
such that $u^n \in J(A)$.  We say that (weak) spectral synthesis fails if there exists a closed subset $A$
of $\Delta(\mathcal{A})$ that is not a set of (weak) spectral synthesis.


\begin{defn}
A multiplier of $\mathcal{A}$ is a linear operator $T$ on $\mathcal{A}$ for which $T(uv) = uT(v)$.
We denote the set of all such maps by $\mathcal{M}(A)$, a Banach space with the operator norm.
\end{defn}

\begin{defn}
Let $\mathcal{X}$ be a Banach $\mathcal{A}$-bimodule.  A derivation of $\mathcal{A}$ on $\mathcal{X}$
is a linear map $D: \mathcal{A} \rightarrow \mathcal{X}$ such that $D(uv) = uD(v) + D(u)v$ for every
$u$, $v \in\mathcal{A}$.
\end{defn}


It is known that $\Delta(A(G)) = G$.  It is also know that when $G$ is compact, and
when $K$ is any closed subgroup, that $\Delta(A(G/K)) = G/K$.  It is shown below
that we need not assume $G$ is compact.
%4.1
\begin{theorem}
$A(G/K)$ is a regular commutative Banach algebra with $\Delta A(G/K) = G/K$.
\end{theorem}
{\it Proof outline:}
Let $\tilde{x}_0 \in G/K$.  $\delta_{\tilde{x}_0} (\tilde{u}) = \tilde{u}(\tilde{x}_0)$ is
continuous multiplicative linear functional on $A(G/K)$.

Suppose $\tilde{\Phi} \in \Delta(A(G/K))$.  We can identify $\tilde{\Phi}$ with $\Phi \in A(G:K)^*$.
$A(G:K)$ is complemented in $A(G)$ by Corollary~\ref{3.4}.  Thus there exists $\Gamma \in VN(G)$
with $P^*_K (\Phi) = \Gamma$ and $\Gamma |_{A(G:K)} = \Phi$.  $\Phi \neq 0$, so $\Gamma \neq 0$.
It can be shown that supp$\Gamma = x_0 K$ for some $x_0 \in G$.  $x_0 K$ is a set of spectral synthesis
for $A(G)$ (Forrest, 1992), therefor $\Gamma$ is the weak-* norm limit of $\Psi = \sum_{i=1}^n a_i L_{x_i}$,
where $x_i \in x_0 K$.  But $L_{x_i} |_{A(G:H)} = \Phi = \delta_{\tilde{x}_0}$.

It can be shown that the map $\tilde{x}_0 \mapsto \delta_{\tilde{x}_0}$ is a homeomorphism
of $G/K$ onto $\Delta(A(G/K))$.\done


We now examine some structural properties of $A(G/K)$.


%4.2
\begin{theorem}\label{4.2}
Let $G$ be a locally compact group with compact subgroup $K$.  The following are equivalent:
\begin{list}{}{\topsep -4pt \itemsep -4pt}
\item[(i)] $G$ is amenable
\item[(ii)] $G/K$ is an amenable coset space
\item[(iii)] $A(G/K)$ has a bounded approximate identity consisting of functions with compact support in $G/K$
\item[(iv)] $A(G/K)$ weakly factorizes
\end{list}
\end{theorem}

%4.3
\begin{cor}
Let $G$ be an amenable locally compact group with a compact subgroup $K$.  Then
${\mathcal M}(A(G/K)) = B(G/K)$ and the usual norms agree.
\end{cor}


Forrest showed that for amenable $G$, every derivation from $A(G)$ into a Banach
$A(G)$-bimodule is continuous.  He extends the result to coset spaces in Theorem~\ref{4.8}
below.

%4.3'
\begin{prop}\label{4.3'}
Let $K$ be a compact subgroup of $G$.  Let $\tilde{E} \subset G/K$ be a set for which (weak)
spectral synthesis fails in $A(G/K)$.  Then (weak) spectral synthesis fails for $q^{-1}(\tilde{E})$
in $A(G/K)$.  In particular, if (weak) spectral synthesis fails for $A(G/K)$, then (weak)
spectral synthesis fails for $A(G)$.
\end{prop}
{\it Proof:}  We give the proof for the case of spectral synthesis.  Assume spectral synthesis
fails in $A(G/K)$ for $\tilde{E}\subset G/K$.  Then there is $\tilde{v}\in I_{G/K}(\tilde{E})$
such that $\tilde{v}\not\in J_{G/K}(\tilde{E})$.  Let $v=\tilde{v}\circ q$.  $v\in I_G (A)$
for $A=q^{-1}(\tilde{E})$.  Suppose now that spectral synthesis holds in $A(G)$ for $A$.
Then $v\in J_G (A)$, and there is a net $\{v_n\} \subset j_G (A)$ such that
$\|v-v_n\|_{A(G)} \rightarrow 0$ as $n\rightarrow\infty$.  Then
$\|P(v-v_n)\|_{A(G)} = \|v-P(v_n)\|_{A(G)} \rightarrow 0$.  $P(v_n) = \tilde{v}_n \circ q$
for some $\tilde{v}\in A(G/K)$, and $\lim_n \|\tilde{v} - \tilde{v}_{n}\|_{A(G/K)} =
\lim_n \|v-P(v_n)\|_{A(G)} = 0$.  Also, $\text{supp}\,P(v_n) \subset (\text{supp}\,v_n)K$.
It follows that $\tilde{v}_n \subset q(\text{supp}\,v_n)$, and that $\tilde{v}_n \in j_{G/K}(\tilde{E})$,
a contradiction since $\tilde{v} \not\in J_{G/K}(\tilde{E})$.\done

%4.4
%\begin{cor}
%Let $G$ be a locally compact group for which $A(G)$ admits (weak) spectral synthesis.
%Then $G$ is totally disconnected.
%\end{cor}

%4.5
\begin{cor}\label{4.5}
Let $G$ be a locally compact group with a compact subgroup $K$.  Then each singleton $\{x\}
\subset G/K$ is a set of spectral synthesis for $A(G/K)$.  Furthermore, if $G$ is amenable,
then every finite subset of $G/K$ is a set of spectral synthesis.
\end{cor}
{\it Proof:}  The first statement follows from Lemma~\ref{4.3'} and from the fact
that $K$ and every coset of $K$ is a set of spectral synthesis for $A(G)$.
If $G$ is amenable, Forrest (1990) showed that any set of the form
$A=\cup_{k=1}^n x_{k}K$ is a set of spectral synthesis for $A(G)$.  Hence, every finite set
in $G/K$ is also a set of spectral synthesis.\done

%4.6
\begin{prop}\label{4.6}
Let $G$ be amenable with compact subgroup $K$.  Let $\{x_1 ,\ldots ,x_n\}$ be a finite
subset of $G/K$.  Then $I = I_{G/K} \{x_1 ,\ldots ,x_n\}$ has a bounded approximate
identity $\{u_\alpha\}$ in $A(G/K) \cap C_c (G/K)$.
\end{prop}
%{\it Proof:}  By Theorem~\ref{4.2}, $A(G/K)$ has a bounded approximate identity
%in $C_c (G/K)$.


%4.7
\begin{theorem}
Let $K$ be a compact subgroup of $G$.  The following are equivalent:
\begin{list}{}{\topsep -4pt \itemsep -4pt}
\item[(i)] $G$ is amenable
\item[(ii)] If $I$ is a cofinite ideal of $A(G/K)$, then $I=I(\{x_1 ,\ldots ,x_n\})$ where $n=\text{codim}\;(I)$
\item[(iii)] Every cofinite ideal in $A(G/K)$ has a bounded approximate identity
\item[(iv)] Each homomorphism of $A(G/K)$ with finite dimensional range is continuous
\end{list}
\end{theorem}


%4.8
\begin{theorem}\label{4.8}
Let $K$ be a compact subgroup of $G$.  The following are equivalent:
\begin{list}{}{\topsep -4pt \itemsep -4pt}
\item[(i)] $G$ is amenable
\item[(ii)] Every derivation from $A(G/K)$ into a Banach $A(G/K)$-bimodule is continuous.
\end{list}
\end{theorem}
{\it Proof:}  $A(G/K)$ is a Silov algebra.  Corollary~\ref{4.5} gives us that each closed
primary ideal in $A(G/K)$ has codimension 1.  Proposition~\ref{4.6} gives us that
each maximal ideal has a bounded approximate identity.  By a result of Bade and Curtis (1994) we
get that each derivation from $A(G/K$ into a Banach $A(G/K)$-bimodule is continuous.

On the other hand, if $G$ is not amenable, $A(G/K)$ does not weakly factorize by Theorem~\ref{4.2}.
$A(G/K)^2$ is not closed in $A(G/K)$ since since it is dense in $A(G/K)$.  Let $\phi$ be some
discontinuous linear functional on $A(G/K)$ with $\phi(u)=0$ for every $u\in A(G/K)^2$.  Let $X$
be a 1-dimensional space, and let $u\cdot x = x\cdot u = 0$ for every $u\in A(G/K)$.
Then the derivation $D: A(G/K) \rightarrow X$ defined by $D(u) = \phi(u)(x)$ is also discontinuous
(Bade, Curtis, 1994).
\done

%% Section 5
%5.1
\section{Weak amenability of $A(G)$}

\begin{defn}
A commutative Banach algebra $\mathcal{A}$ is weakly amenable if every continuous derivation
from $\mathcal{A}$ into a commutative Banach $\mathcal{A}$-bimodule is identically zero.
\end{defn}

$A(G)$ is weakly amenable if $G$ is discrete (Forrest, 1988).  If $G$ is the rotation group
on ${\mathbb R}^3$, then $A(G)$ is not weakly amenable (Johnson, 1994).
When this paper was published, very little was known about the class of groups
$G$ for which $A(G)$ is amenable.  It is here shown that this class contains all totally disconnected
groups.


%5.1
\begin{theorem}\label{5.1}
Let $H$ be an open subgroup of $G$.  Then $A(G/H)$ is weakly amenable.
\end{theorem}
{\it Proof:}  Let $D: A(G/H) \rightarrow \mathcal{X}$ be a continuous derivation into a 
commutative Banach $A(G/H)$-bimodule.  Let $\tilde{u}$ be an idempotent in $A(G/H)$.
Then $D(\tilde{u}) = D(\tilde{u}^n) = nD(\tilde{u})$ for $n\geq 2$.  Thus $D(\tilde{u}) = 0$.
$H$ is open, so the linear span of the idempotents is dense in $A(G/H)$.  Hence $D$ is
identically zero.
\done

%5.2
\begin{lemma}\label{5.2}
Let $G$ be totally disconnected.  Let $u\in A(G)$ and $\epsilon > 0$.  Then there
exists an open compact subgroup $K$ and a $v\in A(G:K)$ such that $\|u-v\|_{A(G)} < \epsilon$.
\end{lemma}
{\it Proof:}  The map $x \mapsto _{x}\!u$, from $G$ into $A(G)$ is continuous.  Therefore
there exists an open neighbourhood $V$ of $e$ such that $x\in V$ implies
$\|u-  _{x}u\|_{A(G)} < \epsilon$.  Let $K$ be an open compact subgroup contained in $V$.
Let $v = P_K (u) = \int_{K}\;_k u\;dk$.  Then
$$
\|u-v\|_{A(G)} = \left\|\int_K (u- _{k}\!u)\;dk\right\|_{A(G)} \leq \int_K \|u - _{k}\!u\|_{A(G)}\;dk
\leq \epsilon.
$$
\done

%5.3
\begin{theorem}\label{5.3}
Let $G$ be disconnected.  Then $A(G)$ is weakly amenable.
\end{theorem}
{\it Proof:}  Let $D: A(G) \rightarrow \mathcal{X}$ be a continuous derivation into a 
commutative Banach $A(G)$-bimodule.  Let $K$ be a compact open subgroup of $G$.  The
restriction of $D$ to $A(G:K)$ determines a derivation of $A(G/K)$.  By Theorem~\ref{5.1},
$D$ is zero on each $A(G:K)$.  By Lemma~\ref{5.2} each $u\in A(G)$ can be approximated
within $\epsilon$ by some $v\in A(G:K)$ for some open compact subgroup $K$.  Thus $D=0$.
\done

This shows that for locally compact totally disconnected group $G$, the span of the
idempotents in $A(G)$ is dense.  Claim: this characterizes totally disconnected groups.
Indeed: the idempotents in $A(G)$ are characteristic functions of open compact subsets
in the coset ring of $G$.  Let $\mathcal{K}$ be the intersection of such open compact
subgroups.  If $G$ is not totally disconnected, $\mathcal{K} \neq \{e\}$.  At the same
time, the idempotents in $A(G)$ are constant of $\mathcal{K}$, and if follows that
their span cannot be dense in $A(G)$.

\pagebreak
%5.4
\begin{prop}\label{5.4}
Let $G_1$, $G_2$ be such that $A(G_i)$ is weakly amenable for $i=1,2$.  Then
$A(G_1 \times G_2)$ is also weakly amenable.
\end{prop}
{\it Proof:}
The projective tensor product $A(G_1)\otimes A(G_2)$ is weakly amenable.
The map $u\otimes v \rightarrow w$, where $w(g_1,g_2) = u(g_1)v(g_2)$ extends to
a continuous homomorphism from $A(G_1)\otimes A(G_2)$ onto a dense subalgebra
of $A(G_1 \times G_2)$.  It follows that $A(G_1 \times G_2)$ is also weakly amenable.
\done

%5.5
\begin{cor}
Let $G=G_1 \times G_2$ where $G_1$ is Abelian and $G_2$ is totally disconnected.  Then
$A(G)$ is weakly amenable.
\end{cor}
{\it Proof:}  $G_1$ Abelian implies $A(G_1)$ is amenable, thus weakly amenable.
Apply Theorem~\ref{5.3} and Proposition~\ref{5.4}.
\done


\end{document}

