\documentclass[12pt]{report}
\pagestyle{plain}
\usepackage{amsmath,amsfonts}

\topmargin -0.5in
\oddsidemargin 0.0in
\textwidth 6.5in
\textheight 9in

\newcommand\done{\begin{flushright}$\Box$\end{flushright}}

\title{Zippin's Theorem}
\author{presented by Alan Kydd}
\date{April 1 2004}

\newtheorem{defn}{Definition}
\newtheorem{theorem}[defn]{Theorem}
\newtheorem{lemma}[defn]{Lemma}

\begin{document}
\maketitle

\begin{defn}
A basis $\{x_j\}$ for a Banach space $X$ is said to be equivalent to a basis
$\{y_j\}$ for a Banach space $Y$ if there exists constants $C_1$,
$C_2 \geq 0$ such that for any $m$ and for any real sequence of scalars $\{a_j\}_{j=1}^m$,
$$
\frac{1}{C_1}\left\|\sum_{j=1}^m a_j y_j \right\| \leq \left\|\sum_{j=1}^m a_j x_j \right\|
\leq C_2 \left\|\sum_{j=1}^m a_j y_j \right\|.
$$
That is, the convergence of $\sum_{j=1}^\infty a_j x_j$ is equivalent to the convergence of
$\sum_{j=1}^\infty a_j y_j$.
\end{defn}

\begin{defn}
The sequence $\{z_j\}_{j=1}^\infty$ is called a block basis with respect to the basis
$\{x_j\}_{j=1}^\infty$ if for every $j$, $z_j = \sum_{i=p_{j}+1}^{p_{j+1}} a_i x_i$, where
$\{p_j\}_{j=1}^\infty$ is an increasing sequence of non negative integers.
\end{defn}

\begin{defn}
A basis $\{x_j\}$ is called unconditional if there exists constant $C>0$ such that for every
sequence of scalars $\{a_j\}$, and for every sequence $\{\theta_j\}$,
where $\theta_j = \pm 1$, $\|\sum \theta_j a_j x_j\| \leq C\|\sum a_j x_j\|$.
\end{defn}

%\begin{defn}
%A basis $\{x_j\}$ is called perfectly homogeneous if it is normalized and every
%normalized block basis $\{z_j\}$ with respect to $\{x_j\}$ is equivalent to
%to the basis $\{x_j\}$.
%\end{defn}

\begin{defn}
A normalized basis $\{x_j\}$ is said to satisfy condition (A) for any two disjoint finite
sets $U$ and $V$ of positive integers and $|s| \leq |t|$, then for every real $\{a_j\}$, $j\in U\cup V$,
$$
\left\| \sum_{j\in U} a_j x_j + s\sum_{j\in V} a_j x_j\right\| \leq
\left\| \sum_{j\in U} a_j x_j + t\sum_{j\in V} a_j x_j\right\|.
$$
\end{defn}

\begin{theorem}\label{daybook}
Let $\{x_j\}$ be an unconditional basis for Banach space $X$.  Then $\{x_j\}$ satisfies condition (A).
\end{theorem}
{\it Proof:}
See~\cite{Day} pp. 67, 68, 73.\done

\begin{lemma}
Let $\{x_j, f_j\}$ denote the biorthogonal sequence of the basis $\{x_j\}$.  If $\{x_j\}$
satisfies condition (A), then $\|f_j\|=1$ for every $j$.
\end{lemma}
{\it Proof:}
$\|f_j\| \geq f_j(x_j) = 1.$  On the other hand,
$$
\|f_j\| = \sup_{\|\sum_{i=1}^\infty a_i x_i\| \leq 1} \left|f_j\left(\sum_{i=1}^\infty a_i x_i\right)\right|
= \sup_{\|\sum_{i=1}^\infty a_i x_i\| \leq 1} |a_j| \leq 1,
$$
since by condition (A), $|a_j| = \|a_j x_j\| \leq \|\sum_{i=1}^\infty a_i x_i \| \leq 1$.\done

\begin{lemma}\label{ntimes}
If $\{x_j\}$ is a basis in a Banach space which satisfies condition (A) and if
$|s_j| \leq |t_j|$ for $1\leq j \leq n$, then $\|\sum_{j=1}^n s_j x_j \| \leq
\|\sum_{j=1}^n t_j x_j \|$.
\end{lemma}
{\it Proof:}
Apply condition (A) $n$ times.\done
\pagebreak
\begin{lemma}\label{c_o}
Let $\{x_j\}$ be a normalized basis in a Banach space $X$ which satisfies condition (A).
If for some $M\geq 1$, $\|\sum_{j=1}^n x_j\| \leq M$ for every $n$, then $\{x_j\}$ is
equivalent to the unit vector basis of $c_o$.
\end{lemma}
{\it Proof:}
By Lemma~\ref{ntimes},
$$
\max_{1\leq j \leq n} |a_j| \leq \left\|\sum_{j=1}^n a_j x_j \right\| \leq
\left(\max_{1\leq j\leq n} |a_j| \right)\cdot\left\|\sum_{j=1}^n x_j \right\| \leq
M\cdot\max_{1\leq j\leq n} |a_j|.
$$
Hence, $\sum_{j=1}^\infty a_j x_j$ converges if and only if $a_j \rightarrow 0$.\done

\begin{lemma}\label{uncond}
Let the basis $\{x_j\}$ be equivalent to all of its normalized block bases.  Then $\{x_j\}$ is
an unconditional basis for $X$.
\end{lemma}
{\it Proof:}
Notice that for any sequence $\{\theta_j\}$, where $\theta_j = \pm 1$, $\{\theta_j x_j\}$ is
a normalized block basis of $\{x_j\}$.  An application of the Banach-Steinhaus theorem
completes the proof.\done

\begin{theorem}
Let $X$ be a Banach space with normalized basis $\{x_j\}$.
$\{x_j\}$ is equivalent to every normalized block basis of $\{x_n\}$ if and only if
$X$ is isomorphic to $c_0$, or to $l_p$ for some $1\leq p < \infty$.

\end{theorem}
{\it Proof:}
The {\it if} part is clear, since the unit vector bases in $c_0$ and $l_p$ are
equivalent to each of their normalized block bases.

Assume that the normalized basis $\{x_j\}$ is equivalent to all normalized block bases.
By Lemma~\ref{uncond} $\{x_j\}$ is unconditional.
By Theorem~\ref{daybook}, $\{x_j\}$ satisfies condition (A).

Next, using a uniform 
boundedness argument similar to the one used for the proof of Lemma~\ref{uncond},
the following is proved:  there exists a constant $M$ such that for every 
normalized block basis $\{u_j\}$ of $\{x_j\}$,
the operator $T_u$ which exhibits the equivalence of the basic sequence (i.e. 
$T_u x_j = u_j$ for all $j$) satisfies $\|T_u\|,\;\|T_u^{-1}\| \leq M$ or equivalently,
\begin{equation}\label{star}
M^{-1}\left\|\sum_{j=1}^\infty a_j x_j\right\| \leq
\left\|T_u\left(\sum_{j=1}^\infty a_j x_j\right)\right\| 
= \left\|\sum_{j=1}^\infty a_j u_j\right\| \leq M\left\|\sum_{j=1}^\infty a_j x_j\right\|.
\end{equation}
for all choices of scalars $\{a_j\}$ such that $\sum_{j=1}^\infty a_j x_j$ converges.

We will give the details of the argument.
Indeed, let $I$ denote the set of all normalized block basis with respect to $\{x_j\}$.
For each $\{z_j\} \in I$,
we denote by $T_z$ the operator defined by $T_z(\sum_{j=1}^\infty a_j x_j) = \sum_{j=1}^\infty a_j z_j$.
We will show that the set $\{\|T_z\| : z\in I\}$ is finitely bounded.
Assume the contrary.  By the theorem of
Banach and Steinhaus, there is an $x = \sum_{j=1}^\infty b_j x_j \in X$ with $\|x\| = 1$ such that
$\sup_{z\in I} \{\|T_z x\|\} = \sup_{z\in I} \{\|\sum_{j=1}^\infty b_j z_j\|\} = \infty$.
Set $n_0 = 0$.
Then we can find a $z^{(1)} = \{z^{(1)}_j\} \in I$ such that
$\|\sum_{j=n_0 + 1}^\infty b_j z^{(1)}_j\| \geq 2$, 
and we can pick an $n_1 > n_0$ large enough so that
$$
\left\|\sum_{j=n_0 +1}^{n_1} b_j z^{(1)}_j \right\| \geq 1.
$$
Let $z_j = z^{(1)}_j$
for $n_0 +1\leq j\leq n_1$.  $\|\sum_{j=n_0 +1}^{n_1} b_j z_j \| \leq
\sum_{j=n_0 +1}^{n_1} |b_j| = A_1$.
Notice that for each $z^{(k)} \in I$,
$$
z^{(k)}_j = \sum_{i=p_j^{(k)} +1}^{p_{j+1}^{(k)}} a_i x_i,
$$
where each $\{p_i^{(k)}\}$ is a sequence of positive increasing integers.
We can find a $z^{(2)} \in I$ such that $\|\sum_{j=n_0 +1}^\infty b_j z^{(2)}_j\| \geq A_1 + 2$ with
$p^{(1)}_{n_1 +1} < p^{(2)}_{n_1 +1} + 1$,
and also such that for some $n_2 > n_1$,
$$
\left\|\sum_{j=n_0 +1}^{n_2} b_j z_j^{(2)} \right\| \geq A_1 + 1.
$$
Let $z_j = z^{(2)}_j$ for $n_1 + 1 \leq j \leq n_2$.  Then we have
$$
\left\|\sum_{j=n_1 +1}^{n_2} b_j z_j \right\| \geq
\left\|\sum_{j=n_0 +1}^{n_2} b_j z_j^{(2)} \right\|
- \left\|\sum_{j=n_0 +1}^{n_1} b_j z^{(2)}_j \right\| \geq
A_1 + 1 - \left\|\sum_{j=n_0 +1}^{n_1} b_j z^{(2)}_j \right\|\geq 1.
$$
Continuing in this fashion we have $\{z^{(i)}\} \subset I$ and $\{n_i\}$ a sequence of increasing positive integers
such that
\begin{equation*}
z_j = 
\begin{cases} z^{(1)}_j & \text{for $n_0 +1 \leq j\leq n_1$}\\
z^{(2)}_j & \text{for $n_1 +1 \leq j \leq n_2$}\\
z^{(3)}_j & \text{for $n_2 +1 \leq j \leq n_3$}\\
& \vdots
\end{cases}
\end{equation*}
\begin{equation}\label{converge}
\left\|\sum_{j=n_i +1}^{n_{i+1}} b_j z_j \right\| \geq 1\quad \text{for all $i\geq 0$}
\end{equation}
\begin{equation}\label{block}
p_{n_i + 1}^{(i)} < p_{n_i + 1}^{(i+1)} + 1 \quad\text{for all $i\geq 1$.}
\end{equation}

By~\eqref{block} the sequence $\{z_j\}$ is a normalized block basis with respect to $\{x_j\}$.
By~\eqref{converge} $\sum_{j=1}^\infty \sum_{i=n_j +1}^{n_{j+1}} b_i z_i$ does not converge.
Since $\{x_j\}$ is an unconditional basis,
$\sum_{j=1}^\infty \sum_{i=n_j +1}^{n_{j+1}} b_i x_i$ does converge, contradicting the
equivalence of $\{x_j\}$ to $\{z_j\}$.  Thus $\{\|T_z\| : z\in I\}$ is finitely bounded.

The finite boundedness of the set $\{\|T^{-1}_z\| : z\in I\}$ is shown is a similar fashion.
This establishes~\eqref{star}.

Taking $u_j = x_{m_j}$ and
\begin{equation*}
a_j =
\begin{cases}
1 & \text{if $1\leq j \leq n$} \\
0 & \text{if $j>n$}
\end{cases}
\end{equation*}
in~\eqref{star} for some increasing sequence $\{m_j\}$ of positive integers, we get
\begin{equation}\label{star-star}
M^{-1}\left\|\sum_{j=1}^n x_j\right\| \leq \left\|\sum_{j=1}^n x_{m_j}\right\|
\leq M\left\|\sum_{j=1}^n x_j\right\|,\;n=1,2,\ldots
\end{equation}


Now, we construct blocks $u_j,\;j=1,2,\ldots$ in the following way: we fix integers $n$ and $k$ and we 
take
\begin{eqnarray*}
u_1 & = & \frac{x_1 + \ldots + x_{n^{k-1}}}{\|x_1 + \ldots + x_{n^{k-1}}\|}, \\
u_2 & = & \frac{x_{n^{k-1}+1} + \ldots + x_{2n^{k-1}}}{\|x_{n^{k-1}+1} + \ldots + x_{2n^{k-1}}\|}\\
    &\vdots
\end{eqnarray*}
We will use the following notation hereafter:
$$
\lambda(n) = \left\|\sum_{j=1}^n x_j \right\|,
$$

By applying (\ref{star}) for these blocks $\{u_j\}$ and suitably chosen $\{a_j\}$, we have that
$$
M^{-2}\lambda(n)\lambda(n^{k-1}) \leq \lambda(n^k) \leq M^2 \lambda(n)\lambda(n^{k-1}),\;n,k=1,2,\ldots
$$
It follows easily by induction on $k$ that
\begin{equation}\label{lambda}
M^{-2k} \lambda(n)^k \leq \lambda(n^k) \leq M^{2k}\lambda(n)^k.
\end{equation}

For any natural $N$, $n$ and $k$ let $h = h(N,n,k)$ be the non-negative integer for which
$N^h \leq n^k < N^{h+1}$.  Thus, $h\log N \leq k\log n < (h+1)\log N$.

It follows from~\eqref{lambda} and after some easy (but not obvious) calculations that
\begin{equation}\label{log}
\left|\frac{\log\lambda(n)}{\log n} - \frac{\log\lambda(N)}{\log N}\right|
\leq 2\log M \left(\frac{1}{\log N} + \frac{1}{\log n}\right).
\end{equation}
Since $1\leq\lambda(n)\leq n$, we get that the sequence $\{\frac{\log\lambda(n)}{\log n}\}$
converges to a limit $c$ where $0\leq c\leq 1$.

Passing to a limit in~\eqref{log} as $N\rightarrow\infty$, we get
$$
M^{-2}n^c \leq \lambda(n) \leq M^2 n^c,\;n=1,2,\ldots.
$$

If $c=0$ we have $\lambda(n) \leq M^2$, and thus by Lemma~\ref{c_o}, $\{x_n\}$ is equivalent to
the unit vector basis of $c_0$.

If $0<c\leq 1$, set $c=1/p$ and we have
\begin{equation}\label{l_p:1}
M^{-2}n^{1/p} \leq \left\|\sum_{j=1}^n x_j \right\| \leq M^2 n^{1/p}.
\end{equation}

To prove equivalence between $\{x_n\}$ and the unit vector basis of $l_p$, we let $r_j,\;j=1,2,\ldots,J$
be any positive rational numbers and assume that $r_j = k_j/k$ where $k_j$ and $k$ are
positive integers.  It follows from \eqref{l_p:1} that
$$
\left\|\sum_{j=1}^J r_j^{1/p} x_j \right\| = k^{-1/p}\left\|\sum_{j=1}^J k_j^{1/p} x_j \right\|
\geq M^{-2}k^{-1/p}\left\|\sum_{j=1}^J \left\|{\textstyle \sum_{i=1}^{k_j}} x_i \right\|x_j\right\|.
$$
Using again~\eqref{star} with the normalized block basis
\[
\begin{array}{llcllc}
u_1 & = &{\displaystyle \frac{x_1 + \ldots + x_{k_1}}{\|x_1 + \ldots + x_{k_1}\|}}, &
a_1 & = &\|x_1 + \ldots + x_{k_1}\|,\\
u_2 & = &{\displaystyle \frac{x_{k_{1}+1} + \ldots + x_{k_{1}+k_{2}}}{\|x_{k_{1}+1} + \ldots + 
x_{k_{1}+k_{2}}\|}}, &
a_2 & = &\|x_{k_{1}+1} + \ldots + x_{k_{1}+k_{2}}\| \\
    &\vdots
\end{array}
\]
we get that
\begin{eqnarray*}
\left\|\sum_{j=1}^J \left\|{\textstyle \sum_{i=1}^{k_j}} x_i \right\|x_j\right\| &\geq &
M^{-1}\left\|\sum_{j=1}^J a_j x_j \right\| \\
& \geq & M^{-2}\left\|\sum_{j=1}^J a_j u_j \right\| =
M^{-2}\left\|\sum_{i=1}^{\sum_{j=1}^J k_j} x_i \right\| \\
& \geq & M^{-4}\left(\sum_{j=1}^J k_j\right)^{1/p} = M^{-4}k^{1/p}\left(\sum_{j=1}^J r_j\right)^{1/p},
\end{eqnarray*}
where the first inequality follows from~\eqref{star-star}, the second inequality follows
from~\eqref{star} for a finite sequence $\{a_j\}$, the first equality follows from the
definition of $a_j u_j$, and the third inequality follows from~\eqref{l_p:1}.
Thus
$$
\left\|\sum_{j=1}^J r_j^{1/p} x_j \right\| \geq M^{-6}\left(\sum_{j=1}^J r_j \right)^{1/p}.
$$
It follows easily from condition (A) that for any real sequence $\{a_j\}$,
$$
\left\|\sum_{j=1}^J a_j  x_j \right\| \geq M^{-6}\left(\sum_{j=1}^J |a_j|^p \right)^{1/p}.
$$
Similar arguments show that
$$
\left\|\sum_{j=1}^J a_j  x_j \right\| \leq M^{6}\left(\sum_{j=1}^J |a_j|^p \right)^{1/p},
$$
and thus $\{x_j\}$ is equivalent to the unit vector basis of $l_p$.

In both the $c_0$ and the $l_p$ case, the equivalence of $\{x_j\}$ to the unit
vectors of each space induces an isomorphism from $X$ onto $c_0$, $l_p$ respectively.
This is a result if the closed graph theorem.
This completes the proof.\done

\begin{thebibliography}{99}

\bibitem{Day} M. M. Day, {\em Normed Linear Spaces}, Springer-Verlag,
1962.

\bibitem{LT} J. Lindenstrauss and L. Tzafriri, {\em Classical Banach Spaces},
Springer-Verlag, 1973.

\bibitem{paper} M. Zippin, {\em On Perfectly Homogeneous Bases in Banach
Spaces}, Israel J. Math. {\bf 4} (1966), 255--272.

\end{thebibliography}
\end{document}

