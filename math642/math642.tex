\documentclass[landscape]{slides}
\usepackage{amsmath,amsfonts}
\title{The $L^p$-Conjecture and Young's Inequality}
\author{Sadahiro Saeki\\presented by Alan Kydd}
\date{12 December 2003}
\pagestyle{plain}
\begin{document}

\maketitle
\begin{slide}
\textbf{The $L^p$-Conjecture}:
Let $G$ be a locally compact group.  If $L^p$ is closed under convulution for some $p \in (1, \infty)$ (i.e. $f*g \in L^p(G)$ for all $f, g \in L^p(G)$), then G is compact.

\textbf{Young's Inequality for Convolution}:  Let $G$ be a locally compact unimodular group.  Let $p,q$ be real numbers such that $1<p,q<\infty$, and $1/p + 1/q > 1$, and let $r$ be defined by $1/r = 1/p + 1/q - 1$.  Then
\begin{enumerate}
\item $L^p(G) * L^q(G) \subset L^r(G)$, and
\item for $f \in L^p(G)$ and $g \in L^q(G)$, we have
$$\|f*g\|_r \leq \|f\|_p \|g\|_q$$
\end{enumerate}
\end{slide}

\begin{slide}
\textbf{History of the $L^p$-Conjecture}
\begin{list}{}{\topsep 0pt \itemsep -1pt}
\item[1.] $G$ abelian, Zelazko (1961)
\item[2.] $G$ arbitrary with $p>2$, Zelazko (1963) and Rajagopalan (1966)
\item[3.] $G$ discrete with $p\geq2$, $G$ totally diconnected with $p=2$, or $G$ either nilpotent or a semi-direct product of two LCA groups, by Rajagopalan (1963, 1966, 1967)
\item[4.] $G$ solvable and $p>1$, Rajagopalan and Zelazko (1965)
\item[5.] $G$ arbitrary with $p=2$ by Rickert (1968)
\item[6.] $G$ amenable with $p>1$ by Greenleaf (1969)
\item[7.] $G$ arbitrary, with $p\in(1, \infty)$ by Saeki (1990)
\end{list}
\end{slide}

\begin{slide}
\textbf{Notation and Definitions}\\
For $p\in [1, \infty]$, we let $p' = p/(p-1)$ for $p > 1$ and $p'=1$ when $p=\infty$.

For $\lambda$, a left Haar measure on $G$, we write $|A|$ for $\lambda(A)$ whenever $A$ is a Haar measurable subset of $G$.

For any function $f$ on $G$, we define $f^{\vee}$ by $f^{\vee}(x) := f(x^{-1})$.

A function $f$ is called symmetric if $f^{\vee} = f$.

We also define $L^p_s := \{f\in L^p: f^{\vee} = f\}$.
\end{slide}

\begin{slide}
\textbf{Lemma 1.1.}  Let $A$ be a compact subset of the general locally compact group $G$.  Then we have
$$|A|^2|A^{m+n}| \leq |A^4|\cdot|A^m|\cdot|A^n|\mbox{ for } m,n \geq 1.$$

\textbf{Lemma 1.2.}  Let $p,q,r \in [1,\infty]$ be such that $1/p+1/q-1/r \neq 1$.  Suppose that $L^p_s * L^q_s \subset L^r$.  Then $G$ is unimodular, $L^p * L^q \subset L^r$, and there exists constant $0<c_o<\infty$ such that
$$\|f*g\|_r \leq c_o\|f\|_p\cdot\|g\|_q \mbox{ for } f \in L^p, g \in L^q.$$

\textbf{Lemma 1.3.}  Let $p,q,r$ and $c_o$ be as in Lemma 1.2.  Then
$$(|A|\cdot|B|)^{1/p'+1/q'} \leq c^2_o|AB|^{2/r'}$$
for all compact $A, B \subset G$.

\textbf{Theorem 1.}  Suppose that there exists $p\in(1, \infty)$ such that $f * g \in L^p(G)$ for all symmetric $f,g\in L^p(G)$.  Then $G$ is compact.
\end{slide}

\begin{slide}
\textbf{On Young's Inequality}\\
Let $p,q,r \in [0,\infty]$.  If $1/r=1/p+1/q-1$, we have
$$\|f*g\|_r \leq \|f\|_p\max(\|g\|_q,\|g^{\vee}\|_q)\mbox{ for }f \in L^{p}_{+}, g\in L^q_+$$
by Young's Inequality (see Theorem (20.18) of Hewitt and Ross, \emph{Abstract harmonic analysis}, Vol. I).  It is well known that for $s \geq r \geq 1$, $L^r \subset L^s$ when $G$ is discrete, and $L^s \subset L^r$ when $G$ is compact.  Together, the above statements give us that
\begin{list}{}{\topsep 0pt}
\item[1.] $\frac{1}{r} \leq \frac{1}{p} + \frac{1}{q} -1$ implies that $L^p * L^q \subset L^r$ for $G$ discrete, and
\item[2.] $\frac{1}{r} \geq \frac{1}{p} + \frac{1}{q} -1$ implies that $L^p * L^q \subset L^r$ for $G$ compact.
\end{list}
\end{slide}

\begin{slide}
\textbf{Questions}
\begin{list}{}{\topsep 0pt}
\item[1.]  If $\frac{1}{r} < \frac{1}{p} + \frac{1}{q} -1$ and $L^p * L^q \subset L^r$, is $G$ discrete?
\item[2.]  If $\frac{1}{r} > \frac{1}{p} + \frac{1}{q} -1$ and $L^p * L^q \subset L^r$, is $G$ compact?
\end{list}
\textbf{Answers}\\
Both are true for abelian groups, Quek and Yap (1983).\\
The answer to the second question is ``no'', for if $G = SL(2,\mathbf{R})$ and $1 \leq p < 2$, then $L^p * L^2 \subset L^2$, Kunze and Stein (1960).
\end{slide}

\begin{slide}
Let $\|\cdot\|_u$ denote the uniform norm, $\|f\|_u = \sup\{|f(x)|:x\in G\}$.  Define a complete norm $\|\cdot\|_{p,u}$ on $L^p_s \cap C^+_{0}(G)$ by $$\|f\|_{p,u} = \max\{\|f\|_{p},\|f\|_u\}.$$

\textbf{Lemma 2.1.}  Suppose that $p,q,r \in [1.\infty]$, $p>1$ and $G$ satisfies $(L^p\cap C_0)*(L^q\cap C_0) \subset L^r$.  Then $G$ is unimodular, and there exists $0<c_1<\infty$ such that
$$\|f*g\|_r \leq c_1\|f\|_{p,u}\|g\|_{q,u}\mbox{ for } f\in L^p\cap C_0, g\in L^q\cap C_0.$$
If, in addition, $G$ is compact, then $r\geq\max\{p,q\}$.

\textbf{Lemma 2.2.}  Let $G,p,q,r$ be as in Lemma 2.1.  Then we have
$$(|A|\cdot|B|)^{1/p'+1/q'} \leq c^2_1|AB|^{2/r'}$$
for all compact $A, B \subset G$ with $|A|,|B| \leq 1$.
\end{slide}

\begin{slide}
\textbf{Theorem 2.}  Suppose that the noncompact group $G$ is such that given any $\epsilon > 0$, there exists a compact $A \subset G$, with large enough $|A|$, such that
$$\liminf_{n \rightarrow \infty} n^{-1}\log\log|A^{2^n}| < \epsilon.$$
Let $1<p<\infty$.  Then there exists $f \in L^p_s \cap C^+_{0}(G)$ such that
$$f * L^q_s \not\subset L^r$$
for all $r,q \in [1,\infty]$ satifying $\frac{1}{r} > \frac{1}{p} + \frac{1}{q} -1.$

\textbf{Corollary 2.3.}  Let $p,q,r \in [1,\infty]$ and $p>1$.  Suppose that $G$ is an infinite LCA group and that $L^p*L^q \subset L^r$.  Then,
\begin{list}{}{\topsep -0.1in \itemsep -0.1in}
\item[1. ]If $G$ is discrete, then $1/r \leq 1/p+1/q-1$.
\item[2. ]If $G$ is compact, then $1/r \geq 1/p+1/q-1$.
\item[3. ]If $G$ is neither discrete nor compact, then $1/r = 1/p+1/q-1$.
\end{list}

\end{slide}

\end{document}